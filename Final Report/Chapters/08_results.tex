% !TEX root = ../main.tex
\chapter{Results}
\label{Results}

\section*{24 hour simulation}
Five clusters of 5 Prosumer Agents (25 Prosumer Agents, 5 Virtual Agents and 1 Supervisor Agents) were simulated over 24 hours. Each cluster had 2 Prosumers generating using Wind Turbines and 3 Prosumers generating using Photovoltaic cells. With no data about the economic output or social utility of the individual households in Rugaragara being available, the data had to be generated for the simulation. All Prosumer Agents were initialised with a random \textit{Productivity} rating and a random \textit{Social Utility} rating to represent a household's hourly contribution to the economic or social development of the local community.

The general trend was that on average, an Agent which was judged to have contributed more (under the Rescher's Canons of Distributive Justice) would receive more, representing in a higher average satisfaction. As an example, the contribution according to the \textit{Rescher's Canon of Distributive Justice} of three of the Prosumers in \texitit{Cluste0} has been included in the 



