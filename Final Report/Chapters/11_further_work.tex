% !TEX root = ./main.tex
\chapter{Further Work}
\label{Further Work}

So far, the works carried out will be only be applicable for one community in rural Rwanda. It is hoped that work will be carried out to make this applicable to other communities by including more generation types such as biogas, and more types of appliances owned by each house.

In the immediate future, it is hoped that this tool could be used to aid the feasibility study of the Micro-Grid to be implemented in Rugaragara Falls. Should there be time, an energy trading platform will be designed and built based on the model created in this project to provide adequate electricity access to all members of the community. However, the energy trading platform would require a different design and implementation scheme which is not included in this report.

The sections below outlines the various properties each Agent must be able to take and how they could be implemented for the simulation scenario of rural communities in Rwanda. All of the properties must be allowed to exist on the same Agent during simulation.

In addition, the e.quinox Micro Grid is currently undergoing feasibility studies. If time allows, a system could be designed and built for the Micro Grid to allow the trading of electricity between households to guarantee electricity supply for when it is needed. This system however will have a completely different set of requirements, which will be drawn up after the completion of this project.

Consider Diesel Generators' costs 
Consider satisfaction of individual agents for them to partake in this

In the simulation, carry forward unused generation because we have storage. Any excess generation is simply curtailed.