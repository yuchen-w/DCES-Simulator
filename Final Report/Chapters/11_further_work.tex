% !TEX root = ../main.tex
\chapter{Evaluation and Further Work}
\label{Further Work}

The work that has been done so far has been only be applicable for a small rural community in Rwanda. However, with small modifications it has the scope to be developed further to a larger scale to model future smart grid networks in developed nations. Currently, much of the costs surrounding the network set-up, network maintenance have not been included in the model. In addition, reactive power flows and losses due to equipment efficiencies were also not modelled. Future work can improve the accuracy of the model by taking into consideration the afore-stated additional costs and losses. In its current iteration, excess generation is curtailed. This is the correct behaviour in poor rural communities where energy storage facilities are limited. In more advanced economies, Smart Grids will need to be able to store unused generation \cite{IEEE-SmartGrid:2015}, so the Simulator will need to be adapted to carry forward unused generation.

An attempt has been made at simulating a community energy system that has yet to be build in Rwanda, which is capable of fairly allocating energy to end users. However, due to a lack of available data on the economic output, and household information from the Rugaragara area, this data had to be generated for the simulation. More data would be gathered required to accurately implement all 8 \textit{Rescher's Canons of Distributive Justice} and to validate the findings that were stated in the Results section.

In the Simulator's present form, only renewable sources of generation are supported fully, as these were the most prevalent type of generation in the Rugaragara Falls area. However, as renewable generation is not the only form of generation available in many rural areas, it would be more realistic to be able to support Prosumer Agents with diesel generators. Work would need to be done to reflect the cost of providing electricity from non-renewable sources to the Common Pool, and appropriately compensate the Agents who operate those generators with a fair allocation.

The current implementation of the Simulator only takes into account variables at a local level. The Simulator design and implementation can also be further expanded to simulate other variables, bottlenecks and opportunities such as government intervention, grid electricity availability and financial incentives. 

Finally, the Simulator assumes no cheating would happen and does not allow cheating in any way. In reality, people will find ways to free-load off the system and bypass the safeguards in place. The design of a future Simulator should be expanded to allow cheating to take place, and implement methods to discourage cheating.

