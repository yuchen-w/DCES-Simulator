% !TEX root = ../main.tex
\chapter{Further Work}
\label{Further Work}

So far, the work carried out has been only be applicable for small rural communities in developing countries, and has the scope to be developed further to a larger scale to model future smart grid networks in developed nations. Currently, much of the costs surrounding set-up costs, network maintenance costs have not been included. Additional limitations include purely resistive loading and no network losses. Future work can improve the accuracy of the model by taking into consideration these additional costs and losses.

An attempt has been made at fairly allocating energy to end users. More data would be required to implement all 8 Rescher's Canons of Distributive justice. 

Currently, only renewable sources of generation were included, as these were the most prevailent in the area studied. It would be realistic to include diesel generators and their costs in further works.

Currently, only one aspect has been explored in the holonic model. The model employed in this simulation can be further adapted to include other bottlenecks and opportunities such as government intervention, grid electricity and financial incentives. The Agents' behavioral model are very simple, and can be further improved.

In the immediate future, it is hoped that this tool could be used to aid the feasibility study of the Micro-Grid to be implemented in Rugaragara Falls. 

Consider Diesel Generators' costs 
Consider satisfaction of individual agents for them to partake in this

In the simulation, carry forward unused generation because we have storage. Any excess generation is simply curtailed.

Trading platform.