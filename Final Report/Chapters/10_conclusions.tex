% !TEX root = ../main.tex
\chapter{Conclusion}
\label{Conclusions}

Since October, an attempt has been made to understand the concept of a Common Pool Resource, and how electricity can be a form of a Common Pool Resource for remote communities in developing countries. An attempt has been made on building a simulator of a Decentralised Community Energy System in developing countries, which can self-organise and fairly allocate electricity to end-users. The testing and results shows that the attempt has been successful in the following:
\begin{itemize}
	\item Incorporating multiple forms of generation, by allowing generation profiles to be imported in the form of CSV files
	\item Incorporating realistic Generator Models by using measured data
	\item Creating a self-organising holonic system which is also able to allocate available power fairly to all system participants
\end{itemize}

However, there were also a number of features that have could have been implemented to make the model potentially more realistic:
\begin{itemize}
	\item Cheating and Selfishness - The model that has been implemented assumes all parties participating in the micro-grid will provide all of their generation and only appropriate their allocated amount from the system. However, in the real world, people will be selfish and attempt to cheat to benefit themselves  
	\item Non-renewable sources of generation - Non-renewable sources of energy have not been included in this model, as they have a fuel cost, which is difficult to model, and also make it difficult to allocate fairly. 
	\item Cost of the system - In reality, a micro-grid costs money to set-up and maintain and no conductor is completely efficient, leading to losses in the cables, and thus additional costs. None of these have been modeled.
\end{itemize}

% \section{}

\section*{Potential Applications}
One obvious use case is the feasibility study of the Rugaragara Falls micro-grid. In the immediate future, it is hoped that this tool could be used to aid the feasibility study of the micro-grid to be implemented in Rugaragara Falls. Aside from modeling small scale community micro-grids, the model can also be scaled up to model country level smart grids once cable losses, reactive power flow have been incorporated into the simulation. 
This Simulator of a Decentralised Community Energy System can be further developed with more features to explore possible utilisation patterns within rural communities, and also explore controls for regulating the use of electricity in developed countries. 