% !TEX root = ../main.tex
\chapter{Conclusion and Evaluation}
\label{Conclusions}

Since October, an attempt has been made to understand the concept of a Common Pool Resource, and how electricity can be a form of a Common Pool Resource for remote communities in developing countries. An attempt has been made on building a simulator of a Decentralised Community Energy System in developing countries, which can self-organise and fairly allocate electricity to end-users. The testing and results would suggest that the attempt has been successful.

The simulator has been successful in:
\begin{itemize}
	\item Incorporating multiple forms of generation, by allowing generation profiles to be imported in the form of CSV files
	\item Incorporating realistic Generator Models by using measured data
	\item Creating a self-organising holonic system which is also able to allocate available power fairly to all system participants
\end{itemize}

However, there were also a number of features that have no been implemented to make the model potentially more realistic:
\begin{itemize}
	\item Cheating and Selfishness - The model that has been implemented assumes all parties participating in the microgrid will provide all of their generation and only appropriate their allocated amount from the system. However, in the realworld, people will be selfish and attempt to cheat to benefit themselves  
	\item Non-renewable sources of generation - Non-renewable sources of energy have not been included in this model, as they have a fuel cost, which is difficult to model, and also make it difficult to allocate fairly. 
	\item Cost of the system - In reality, a micro-grid costs money to set-up and maintain and no conductor is completely efficient, leading to losses in the cables, and thus additional costs. None of these have been modelled.
\end{itemize}

\section{}

\section*{Potential Applications}
One obvious use case is the feasibility study of the Rugaragara Falls micro-grid. Aside from modelling small scale community micro-grids, the model can also be scaled up to model country level smart grids once cable losses, reactive power flow have been incorporated into the simulation. 
This Simulator of a Decentralised Community Energy System can be further developed with more features to explore possible utilisation patterns within rural communities, and also explore controls for regulating the use of electricity in developed countries. 

\section*{Further Work}

So far, the work carried out has been only be applicable for a small rural community in Rwanda, and has the scope to be developed further to a larger scale to model future smart grid networks in developed nations. Currently, much of the costs surrounding the network set-up, network maintenance have not been included. Additional limitations include purely resistive loading and no network losses. Future work can improve the accuracy of the model by taking into consideration these additional costs and losses.

An attempt has been made at simulating realistically a community, and fairly allocating energy to end users. More data such as household economic output, household utility would be required to implement all 8 Rescher's Canons of Distributive Justice. 

Currently, only renewable sources of generation were included, as these were the most prevailent in the area studied. It would be realistic to include diesel generators and their costs in further works.

The simulator in its present form only one aspect has been explored in the holonic model. The model employed in this simulation can be further adapted to include other bottlenecks and opportunities such as government intervention, grid electricity and financial incentives. The Agents' behavioral model are very simple, and can be further improved.

In the immediate future, it is hoped that this tool could be used to aid the feasibility study of the Micro-Grid to be implemented in Rugaragara Falls. 

Consider Diesel Generators' costs 
Consider satisfaction of individual agents for them to partake in this

In the simulation, carry forward unused generation because we have storage. Any excess generation is simply curtailed.

Trading platform.