% !TEX root = ../main.tex
\chapter{Introduction}
\label{introduction}

Electricity connections in many developed nations are wide-spread and reliable. Residents in the UK benefits from 100\% electricity penetration\cite{World-Bank-web:2015} and a Transmission network which operates above 99\% reliability \cite{NG-web:2015}. In the rural areas of many developing countries, electricity is considered a luxury; there are often no wide-spread access to a continuous and reliable electricity supply in the rural areas of many developing countries \cite{IEA-web:2015}. However there exists many isolated sources of electricity such as solar panels, wind turbines and standalone diesel generators owned by wealthy households, shops and large organisations. 

The aim of this project is to explore the idea of a \ac{dCES} which brings together existing generation infrastructure to provide a reliable electricity supply in rural communities. With the vast majority of people living in communities such as villages, towns and cities, the structure of the decentralised energy system model was designed to be scalable and allows the formation of communities. In a traditional electricity system, electricity largely flows in a uni-directional manner. Electricity is generated away from demand centres, and transported around the country by transmission systems and distributed to consumers by the distribution network. With many of the users of the dCES System capable of both consuming and generating, a scalable micro-grid for this model needs to be designed to be bi-directional.

With the goal of scalability and bi-directional power flow in mind, the model was constructed as a holonic \ac{MAS}. A holonic system by definition is one which is formed of many smaller systems, which are in turn formed of many smaller systems and so on, until reaching the most "elementary" of systems \cite{Pitt:Holonic_Institutions}. Within a holonic MAS, individual consumer-provider households and businesses can be modelled as Agents, which can be grouped together to form larger communities such as villages which can be modeled as a single entity. These entities can be further grouped together to form even larger entities such as districts or provinces within the Simulator. Basing the design of the simulator on that of a holonic system simplifies the scalability aspect of having multiple communities in a simulation. 

A MAS was selected to simulate the dCES the as Agents within a MAS are required to be:
\begin{itemize}
	\item able to act independently
	\item unable to directly manipulate the environment
	\item unable to control the actions of other Agents. 
\end{itemize}
In the case of this simulator, consumers are independent entities who:
\begin{itemize}
 	\item can't change the environment conditions e.g. have access to Grid Electricity, which satisfies the \textit{unable to directly manipulate the environment} condition
 	\item unable to directly control the consumption and provision of other consumers or providers, satisfying the conditions of \textit{able to act independently} and \textit{unable to control the action of other Agents} 
\end{itemize}
Presage 2 was selected as the MAS platform for this project as support was available from PhD students within the EEE department who currently use the platform. However, there are also some limitations with the platform. The simulation results can occasionally produce inconsistent results due to the parallel execution nature of the simulation platform, and the platform is limited to execute simulations in discrete time steps. Suitable design steps have been taken to mitigate these limitations and have been detailed in the Design section of this report. 

To allow the model to be realistic, the actual rural communities in Rwanda with realistic demand and generation profiles were simulated. Due to the lack of available data for some of the Simulation variables, the demand and generation profile has been approximated using data from rural UK demand centres. Results from these simulations indicate the Simulator work as expected.
