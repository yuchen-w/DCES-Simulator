%%%%%%%%%%%%%%%%%%%%%%%%%%%%%%%%%%%%%%%%%
% Maggi Memoir Thesis (WriteLaTeX Version - Compiles with pdflatex)
% XeLaTeX Template
% Version 1.0 (22/12/13)
%
% This template has been downloaded from:
% http://www.LaTeXTemplates.com
%
% Original authors:
% Federico Maggi (fede@maggi.cc) with extensive modifications by:
% Vel (vel@latextemplates.com)
%
% License:
% CC BY-NC-SA 3.0 (http://creativecommons.org/licenses/by-nc-sa/3.0/)
%
% Important note:
% Most of the document content and packages are specified within structure.tex
% so if you need to make modifications to the template have a look there first!
%
%%%%%%%%%%%%%%%%%%%%%%%%%%%%%%%%%%%%%%%%%

%----------------------------------------------------------------------------------------
%   PACKAGES AND OTHER DOCUMENT CONFIGURATIONS
%----------------------------------------------------------------------------------------

\documentclass[11pt,a4paper,twoside]{memoir} % Change font size here (allowable values are 9pt-12pt), change the paper size, specify one or two sided printing and specify whether to show trimming lines

\input{structure.tex} % Include the file containing the code defining the structure and style of the document
%------------------------------------------------
% Fonts

\renewcommand*{\acffont}[1]{{\normalsize\itshape #1}} % Font style for the acronym text (e.g. Do It Yourself)
\renewcommand*{\acfsfont}[1]{{\normalsize\upshape #1}} % Font style for the acronym in bracket (e.g. (DIY))

%------------------------------------------------
% Hyphenations

\hyphenation{a-no-ma-lous a-no-ma-ly amounts breaches} % Specify custom hyphenation points in words with dashes where you would like hyphenation to occur, or alternatively, don't put any dashes in a word to stop hyphenation altogether

%----------------------------------------------------------------------------------------
%   TITLE PAGE
%----------------------------------------------------------------------------------------
%----------------------------------------------------------------------------------------

\makeindex % Write an index file

\begin{document}

%
\begin{titlepage}
                % \newgeometry{top=25mm,bottom=25mm,left=38mm,right=32mm}
                \setlength{\parindent}{0pt}
                \setlength{\parskip}{0pt}
                % \fontfamily{phv}\selectfont

                {
                                \Large
                                \raggedright
                                Imperial College London\\[17pt]
                                Department of Electrical and Electronic Engineering\\[17pt]
                                Final Year Project Report 2015\\[17pt]
 
                }

                \rule{\columnwidth}{3pt}
                \vfill
                \centering
                  \includegraphics[width=0.7\columnwidth,height=60mm,keepaspectratio]{background/cover.png}
                \vfillba
                \setlength{\tabcolsep}{0pt}

                \begin{tabular}{p{40mm}p{\dimexpr\columnwidth-40mm}}
                                Project Title: & \textbf{Proactive Multimedia Caching} \\[12pt]
                                Student: & \textbf{Leon Zhang} \\[12pt]
                                CID: & \textbf{00683680} \\[12pt]
                                Course: & \textbf{EEE4} \\[12pt]
                                Project Supervisor: & \textbf{Dr. Deniz Gunduz} \\[12pt]
                                Second Marker: & \textbf{Dr T.K. Kim} \\
                \end{tabular}


\end{titlepage}

\begin{titlingpage}
                % \newgeometry{top=25mm,bottom=25mm,left=38mm,right=32mm}
                \setlength{\parindent}{0pt}
                \setlength{\parskip}{0pt}
                % \fontfamily{phv}\selectfont

                {
                                \Large
                                \raggedright
                                Imperial College London\\[17pt]
                                Department of Electrical and Electronic Engineering\\[17pt]
                                Final Year Project Report 2015\\[17pt]
 
                }

                \rule{\columnwidth}{3pt}
                \vfill
                \centering
                \includegraphics[width=1\columnwidth,height=100mm,keepaspectratio]{Images/Model2.png}
                \vfill
                \setlength{\tabcolsep}{0pt}

                \begin{tabular}{p{40mm}p{\dimexpr\columnwidth-40mm}}
                                Project Title: & \textbf{Simulator for Decentralised Energy Systems} \\[12pt]
                                Student: & \textbf{Yuchen Wang} \\[12pt]
                                CID: & \textbf{00683700} \\[12pt]
                                Course: & \textbf{EE4 (EM)} \\[12pt]
                                Project Supervisor: & \textbf{Prof. Jeremy V. Pitt} \\[12pt]
                                Second Marker: & \textbf{Dr. Javier A. Barria} \\
                \end{tabular}
\end{titlingpage}


\frontmatter % Use roman page numbering style (i, ii, iii, iv...) for the pre-content pages

%----------------------------------------------------------------------------------------
%   PREFACE
%----------------------------------------------------------------------------------------

\section*{Acknowledgements}
Foremost, I would like to thank my supervisor Prof. Jeremy Pitt for taking the time to impart advice, guidance and supervision over the past academic year on this project. \\ 
Besides my supervisor, I would like to thank e.quinox, all members past and present for providing me with the opportunity of a lifetime and the information without which, would not make this project possible. \\
I would also like to take this opportunity to thank my friends, housemates and fellow colleagues for their continuous encouragement, laughter and the fun times we have shared over the last four years. \\
Last but not the least, I would like to thank my mother for all of her continued support, inspiration and love.

\begin{flushright}
%\textsc{\theauthor}\\
Yuchen Wang\\
June 2015
\end{flushright}

\cleartoverso % Force a break to an even page

%----------------------------------------------------------------------------------------
%   ABSTRACT
%----------------------------------------------------------------------------------------

\begin{abstract}

In many developing countries the electricity grid network is underdeveloped, causing low levels of rural electrification. However, electricity access in these rural communities are not non-existent. There exists a number of isolated and independent electricity generators owned by individuals, NGOs and local government institutions. This project aims to accurately simulate a Decentralised Community Energy System which utilises these isolated sources of electricity to better serve the local communities. \\
\indent This project successfully designed, implemented and tested a Simulator for a Decentralised Community Energy System using a holonic Multi-Agent System, with electricity being contributed to a common pool and allocated fairly to consumers by applying \textit{Rescher's Canons of Distributive Justice}.  \\
\end{abstract}

\cleartoverso % Force a break to an even page


%----------------------------------------------------------------------------------------
%   TABLE OF CONTENTS
%----------------------------------------------------------------------------------------

\tableofcontents* % Print the table of contents

\cleartoverso % Force a break to an even page

%----------------------------------------------------------------------------------------
%   LIST OF FIGURES
%----------------------------------------------------------------------------------------

\listoffigures % Print the list of figures

\cleartoverso % Force a break to an even page

%----------------------------------------------------------------------------------------
%   LIST OF TABLES
%----------------------------------------------------------------------------------------

\listoftables % Print the list of tables

\cleartoverso % Force a break to an even page

%----------------------------------------------------------------------------------------
%   ACRONYMS
%----------------------------------------------------------------------------------------

\include{acronyms} % Include a List of Acronyms section using acronyms.tex where they are defined

%\cleartoverso % Force a break to an even page

%----------------------------------------------------------------------------------------
%   COLOPHON
%----------------------------------------------------------------------------------------

% \thispagestyle{empty} % Remove all headers and footers from this page

% \vspace*{2em}
% \renewcommand{\abstractname}{Colophon}
% \begin{abstract}
% This document was typeset using the \textsf{XeTeX} typesetting system created by the Non-Roman Script Initiative and the memoir class created by Peter Wilson. The body text is set 10pt with~Adobe Caslon Pro. Other fonts include \texttt{Envy Code R}, \textsf{Optima Regular} and. Most of the drawings are typeset using the \textsf{TikZ/PGF} packages by Till Tantau.
% \end{abstract}
% \vfill

%----------------------------------------------------------------------------------------
%   CONTENT CHAPTERS
%----------------------------------------------------------------------------------------

\mainmatter % Begin numeric (1,2,3...) page numbering

\chapterstyle{thesis} % Change the style of the Chapter header to that defined in structure.tex

\pagestyle{Ruled} % Include the chapter/section in the header along with a horizontal rule underneath

% !TEX root = ../main.tex
\chapter{Introduction}
\label{introduction}

Electricity is in wide-spread use in many developed countries and are delivered by a vast network of cables, overhead lines and other assets maintained by Distribution and Transmission Network Operators. Electricity access has 100\% penetration in the UK \cite{World-Bank-web:2015}, and residents also benefits from a Transmission Network which operates at above 99\% reliability \cite{NG-web:2015}. In many areas of rural developing countries, there are often no wide-spread access to a continuous and reliable electricity supply which is provided by the country's electricity grid \cite{IEA-web:2015}. However there exists many isolated sources of electricity generation such as solar panels and standalone diesel generators utilised by relatively wealthy households, shops and buildings belonging to large organisations. 

The aim of this project is to explore the idea of using a holonic institution to model a Decentralised Community Energy System which bring together the existing decentralised generation infrastructure to provide a reliable electricity for users in rural communities. The model was constructed as a Multi-Agent System (MAS) and simulated using Presage 2 with individual consumers such as households and providers such as generators modelled agents. The agents can be grouped together to form communities such as villages, which in turn can be grouped together to form a larger entity such as a district or a province. 

With the vast majority of people living in communities such as villages, towns and cities, the structure of the decentralised energy system model will need to be designed in a fashion that is scalable and allows grouping of people to form communities. In a traditional electricity system operated by most developed countries is composed of centralised generation, transported around the country by transmission systems and distributed to consumers by the distribution network. Traditional models and structures do not apply to this project as they have been designed for mostly uni-directional power flow: from large generators to consumers. With many of the users of the Decentralised Community Energy System capable of both consuming and generating, a scalable micro-grid for this model needs to be designed with that in mind.

Designing the structure of the simulation on the design of holonic systems simplifies the scalability aspect of having multiple communities in a simulation. A holonic system is one which is formed of many smaller systems, which are in turn formed of many smaller systems and so on, until reaching the most "elementary" of systems. In the case of this project, the simulation will be designed to be able to distribute fairly electrical power between interrelated agents which are in turn composed of interrelated subagents recursively, until reaching lowest level of subagents (households and businesses). 

A Multi-Agent System was selected to simulate as Agents are required to be able to act independently, to be unable to directly manipulate the environment and unable to control the actions of other Agents. In the case of this simulator, consumers are independent entities who can't change the environment conditions (e.g. have access to Grid Electricity) and unable to directly control the consumption and provision of other consumers or providers. Presage 2 was selected as the Multi-Agent Simulation platform as it was possible to seek support from PhD students within the EEE department who currently use the system.

With Presage 2, there are also some issues and limitations. The simulation results can sometimes be inconsistent due to out of order parallel execution by the simulation platform, and the simulation can only execute in discrete time steps. Suitable design steps have been taken to mitigate these problems and limitations which have been detailed in the Design section of this report. 

To allow the model to be realistic, the model has been based on real rural communities in Rwanda, with realistic demand and generation profiles. Due to the lack of available data in Rwanda, the demand and generation profile has been approximated using data from rural UK load centres.

This report outlines the design, implementation and testing of a holonic multiagent simulation of a decentralised energy system. 
 % Include the introduction chapter
% !TEX root = ../main.tex
\chapter{Background}
\label{Background}

\section*{Electricity as a Common Pool Resource}
A Common Pool Resource is a depletable resource which can be utilised by a group of people, characterised by a reduction in the availability of this resource as individuals withdraw or utilise this resource \cite{Ostrom:90}.  Electricity can be a Common Pool Resource if there exists a finite amount of electricity generation capacity. As users connect demand appliances to the generators, the availability of electricity supply for additional demand diminishes.

In developing communities with significant generation from renewable sources such as wind and solar, the availability of power can be highly variable between periods in time. This inherent volatility in the amount of available resource means that many communities can not be self-sustainable. By connecting multiple households and communities of consumer-providers (or \textit{Prosumer}) together, electricity generation and consumption can become diversified, increasing the sustainability of electricity access. Electricity in this case becomes a highly volatile Common Pool Resource.

Ostrom showed that Common Property Regimes (arrangements which resource consumers agree to) can be formed to maintain the Common Pool Resources by controlling the access to the resource \cite{Ostrom:90}. For this project, the arrangement will be the participation in a local micro-grid between a number Prosumers. The micro-grid will provide the infrastructure to allow electricity to be pooled as a Common Pool Resource, and will also provide the means to control the access to the resource.

\section*{Distributive Justice and Fair Allocation}
For the purpose of this project, a Micro-Grid is assumed to already exist and is being operated automatically by a third party. The Micro-Grid is also assumed to be able to enforce Prosumer contribution to the Common Pool and restrict its access. With the infrastructure for enforcing the \ac{CPR}, it is important to ensure that the allocation would be fair to encourage Prosumers to be part of the system. Being fair forms two of the necessary Ostrom's Principles for Enduring Institutions in a CPR \cite{Ostrom:90}. \\

\section*{Multi-Agent Simulation}
The simulation will be designed as a Multi-Agent System (MAS). MASes are particularly suited for this kind of model as Agents in MASes have three very important characteristics:
\begin{itemize}
	\item Autonomy: Agents act independently
	\item Local view: no Agent can see or manipulate the environment it is in
	\item Decentralised: no Agent can control the action of all Agents
\end{itemize}
In reality, individual households which are represented by Agents in the simulator all perform actions according to their individual and unique needs, and are not controlled by a third party. This makes Autonomy and Decentralisation a requirement for the Agents in the Simulator. Participants or households connected to the network can't directly control how other participants use or generate electricity for the Common Resource Pool, making it an requirement for the Agent to have a Localised view. 

\section*{Decentralised Community Energy Systems as a Holonic System}
A holonic system (or holarchy) is a system which is composed of interrelated subsystems or institution, each of which are in turn composed of "sub-subsystems"  or institutions and so on, recursively until reaching a lowest level of "elementary" subsystems. Each system, sub-system or institution has a well-defined set of goals or objectives which is achieved through enforcing a set of rules on its members or member-systems\cite{Pitt:Holonic_Institutions}. A holonic MAS is what will be utilised to simulate the dCES and associated CPR which allows the simulator to be scalable with any number of community participants and any number of communities. 

In the context of a rural dCES, a holonic system in this case would be composed of communities such as Districts, Provinces, Sectors which are composed of "sub-communities" such as Towns and Villages. The sub-communities would be composed of many "elementary" subsystems such as households, businesses and other points of connection for electricity. Each community or institution has the goal of gathering all generation from members, and subsequently fairly allocating electricity to all members. This goal would be achieved with the assumption that they are provided with the necessary infrastructure and powers for enforcing quotas and contribute to a common pool of electricity.

\subsection*{About Presage 2}
Presage 2 is a simulation platform for multi-nodal or Agent simulation of societies. The platform was built by Sam Macbeth and is currently maintained by PhD students within Imperial. This platform was chosen for the simulator as it enables the investigation of the impact of agent design (such as household behaviour), network properties (constraints on access) and the physical environment on individual agent behaviour and long-term global network performance \cite{Presage2-Desc:2015}. In the context of this Project, each Node or Agent can represent individuals, households, businesses or generators. 

In Presage 2, Agents are only allowed to act during increments of time steps, which makes the simulation a discrete time driven one. During each time step, all actions are required to be performed by Agents via \textit{Action Handlers} and \textit{Environment Services}. Figures \ref{fig:Presage_architecture} and \ref{fig:Presage_sim_architecture} illustrates the general and runtime architecture.

\begin{figure}[h!]
	\centering
	\includegraphics[scale=0.25]{Images/Presage.jpg}
	\caption{General Presage 2 architecture \cite{Presage_Kyoto:2015}}
	\label{fig:Presage_architecture}
\end{figure}

\begin{figure}[h!]
	\centering
	\includegraphics[scale=0.4]{Images/Presage2.jpg}
	\caption{Presage 2 simulation architecture \cite{Presage_Kyoto:2015}}
	\label{fig:Presage_sim_architecture}
\end{figure}

% \clearpage


% !TEX root = ../main.tex
\chapter*{Analysis}
\label{Analysis}

\section*{Decentralised Generation in Rural Communities of Rwanda}
With an estimated 25\% rural electrification rate in 2009 \cite{IEA-web:2015}, vast amounts of rural communities in Africa remain un-electrified. Benefits of a localised local micro-grid would be felt immensely rural Africa, making rural Africa is one obvious candidate for simulation scenarios. For a realistic simulation, knowledge of existing infrastructure in place will need to be obtained. With many possible countries to choose from in Africa, Rwanda in particular has been identified as a good potential simulation scenario to research. Data about electricity consumption is often difficult to source for rural communities in developing countries. Fortunately, some research data are available from the student society e.quinox for Rwanda. e.quinox is a student-led society which aims to find a scalable solution for rural electrification who mainly operate in Rwanda \cite{e.quinox-web:2015}. The subsections below outline some of the ways remote rural communities are able to access electricity in Rwanda.

\subsection*{Electricity Generation}
One of the solutions currently being implemented by e.quinox is the "Energy Kisok model" \cite{e.quinox-EK-web:2015}. The "Energy Kiosk" model features an Energy Kiosk - a building where the generation, storage and distribution of electricity takes place. In e.quinox operated kiosks, electricity is generated from renewable sources, stored on-site and distributed via leased storage devices. Traditionally, generation have come from roof mounted solar panels. More recently however, hydro-electric generation has been demonstrated to be feasible with the construction of a "Hydro Kiosk" at Rugaragara Falls in Southern Rwanda completed in 2012 \cite{e.quinox-Hydro-web:2015}.

\subsection*{Storage and Distribution}
Within each kiosk, electricity that is generated is stored in storage batteries placed within the kiosks. The storage batteries regulate power output and allows access to electricity in the kiosk even during periods of no electricity generation. In the absence of any electricity distribution infrastructure, e.quinox has traditionally leased a number of portable batteries to the local population for the purpose of electricity distribution. An example of the portable batteries leased can be seen in figure \ref{fig:AmaziBox}.

\begin{figure}[h!]
\centering
\includegraphics[scale=0.5]{Images/AmaziBox.jpg}
\caption{First Generation e.quinox Battery Boxes deployed at Rugaragara Falls Kiosk}
\label{fig:AmaziBox}
\end{figure}

Consumers from the local community pay to hire the battery boxes under one of two payment schemes: pay-per-recharge and pay-per-month \cite{e.quinox-Hydro-web:2012}. The biggest difference between the two schemes are that users can recharge as often as they would like with pay-per-month. Both payment schemes involve the recharge of the boxes at the energy kiosk when they are depleted of energy.

\subsection*{Standalone Solution}
The Standalone Solution is an independent electrification solution which was recently developed by e.quinox for customers who live far from energy kiosks.

The Standalone solution consists of a pay-as-you-go solar electricity generation and storage kit, known as the Izuba.Box \cite{e.quinox-Standalone-web:2012}.  With the Izuba.Box, customers no longer have to travel regularly back to the Energy Kiosk for electricity. Solar panels are installed on the customer's roof, and is connected to a sealed box which contains a large battery box. The attached large battery box allows a regulated power output and access to electricity during dark hours. 

\begin{figure}[h!]
\centering
\includegraphics[scale=3]{Images/standalone_box.jpg}
\caption{First Generation Izuba.Box deplyed in Minazi, Northern Rwanda}
\label{fig:IzubaBox}
\end{figure}

With the customers not returning to Energy Kiosks, the battery boxes are not hired out like the battery boxes are. The high capital costs of the independent solar system is spread over a typically two year rent-to-own payment plan using a mobile payment system.

It is hoped that the Standalone Solution and additional generation from the Energy Kiosk can be complement the battery boxes in circulation to provide a continuous access to electricity to all households in the village.

\subsection*{Micro-Grid}
With the recent completion of a hydro-electric kiosk, e.quinox for the first time has a kiosk with access to an always-on generator. However, with a limited number of battery boxes in circulation, the constant water flow available during the off-peak hours leaves an excess in generation capacity. To improve utilisation of the generator, e.quinox has recently started conducting a feasibility study into constructing a transmission line and a distribution network which will serve communities near the kiosk.

Preliminary surveys conducted in the nearest villages to the kiosk indicates the demand could exceed the amount of excess power generated by the kiosk, making it an ideal scenario for this project. The result of this project can be used in conjunction with other e.quinox feasibility studies to implement a novel type of Micro-Grid with compulsory demand side management. 

\section*{Rugaragara Falls as a Simulation Scenario}
In developing countries such as Rwanda, poor communities with no access to grid electricity are often in isolated locations such as Rugaragara Falls. The local District Sector office estimates that Grid electricity access won't be available in Rugaragara before 2020. In the case of Rugaragara Falls and much of rural Rwanda, the difficult terrain and underdeveloped transport links make fuel expensive to obtain. Therefore instead of non-renewable sources of generation, locals often depend on renewable sources of electricity such as solar and wind. However, renewable generation can be highly variable in the amounts of electricity that is produced due to external factors such as weather, time of day and the season. Without access to redundancies received from the national electricity grid, it can be beneficial for Prosumers within these areas to form a micro-grid to reduce the generation requirement of individual households for continuous electricity access.

%As an example, many people still choose to use battery boxes provided by e.quinox, which is cheaper than connecting to the grid.

% \subsection{Additional Research}
% e.quinox in Rwanda only presents us with one scenario. Additional research is required for other regions such as South America and the Indian Subcontinent to look at potential new scenarios and use cases for various generators and electrical appliances.

% The information above only outlines the existing infrastructures in place. Additional work needs to be done to obtain relevant usage data from e.quinox customers and other electricity users to accurately build a relevant generation mix and demand model.

% In some regions such as Minazi, e.quinox provides the only source of accessible electricity for the local populace. Data on their usage should be more readily available than some of the other locations in other countries. 
% !TEX root = ../main.tex
\chapter{Requirements and Design}
\label{Requirements and Design}
In this project, a model of a rural electricity network that is disconnected to the grid is expected to be modeled. To allow the network to be scalable, the network is holonic in structure and consist of communities of "smart households" which can generate and utilise electricity depending on their demand and generation profiles. Therefore the simulator was required to have the following features:
\begin{itemize}
  \item Multiple Forms of Generation: Renewable and Non-renewable generators which can operate continuously or discontinuously
  \item Realistic Generator Models: Programmable variable generation power output to simulate wind and solar power
  \item Multiple demand centres: the simulation will be of one or more communities operating with a number of households/businesses requiring electricity
  \item Self organising by the system to allocate the available power fairly to all users
  \item Presage 2: The simulation will be programmed in Java using Presage 2.
\end{itemize}

As this was a simulation implemented in Presage 2, there were no specific requirements which must be adhered to with regards to speed, portability and performance.  

The simulation was developed using Presage 2, with the hope that a network of Decentralised Community Energy Systems can be simulated, and an algorithm for fairly allocating available generation to demand will be implemented. It is assumed that no cheating will take place.

\section*{Model Design}
To allow communities to work as holonic systems, communities will be created and simulated akin to the simplified model in Figure \ref{fig:SimpleModel}. The Agents in this case are reprsented by the Circles labeled A-H, with various demand/generation equipment connected to the Agents. The Agents are connected to a Virtual Agent or a Parent Agent represented by the single black dot that all Agents on the periphery are connected to in Figure \ref{fig:SimpleModel}.

\clearpage
\begin{figure}[h!]
	\centering
	\includegraphics[scale=0.8]{Images/Model.png}
	\caption{A simplified model diagram}
	\label{fig:SimpleModel}
\end{figure}

Agents that group together are connected to a central virtual agent to allow the agents to form a community. These communities can further connected to another virtual agent to form even larger communities demonstrated in Figure \ref{fig:SimpleModel2}. 

\begin{figure}[h!]
	\centering
	\includegraphics[scale = 0.9]{Images/Model2.png}
	\caption{A simplified network model diagram}
	\label{fig:SimpleModel2}
\end{figure}


\subsection*{Design and Model Assumptions}
To simplify the implementation of the model, a number of assumptions will be made in the design process:
\begin{itemize}
	\item No losses would be incurred by the network
	\item All load on the network will be purely resistive
	\item All generation will act as negative load
	\item Only basic appliances such as phones, lights and fridges will be connected to the vast majority of households. 
\end{itemize}

\subsection*{Types of Agents}
Presage's main components are Agents, which are the actors which can act on the environment.
\subsubsection*{Virtual Agents}
Virtual Agents represents all connected sub-communities or Agents in community of Virtual Agents. These Agents will be responsible for enforcing quotas on connected sub-communities or Agents and collecting Generation for the Common Pool. 

\subsubsection*{Prosumer Agents}
Agents represent the most "elementary" system such as generators, households, businesses and other demand centres.

\subsection*{Agent Properties}
Agent Properties represent the information that is to be relayed to the Community or Institution that the Agent is part of.
\paragraph*{Demand} - All Agents will also have the Demand property which represents aggregated electricity consumption at a point of connection. Assumed demand curves will be produced from survey data of potential customers in the area for the initial testing. Should the survey data not be available for the area, a reasonable approximation will be made based a predicted usage habits of the wider local population.

It is anticipated that the final simulation will have a desired demand profile that each agent will aim to have. 

For Virtual Agents, this property represents the aggregated Demand of all connected sub-communities or Agents and will not have any associated Demand profiles.
% However, the change in demand profile due to appropriation of energy must be able to satisfy the expected behavioral habits of the local inhabitants. These habits should include restrictions such as no trading during hours of sleep for households. What the habits will be, and how this will be implemented will be determined after some additional research into the area.

\paragraph*{Generation} - All Agents will have a Generator property which will allow all agents to generate power. For Virtual Agents, this property represents the aggregated Generation of all connected sub-communities or Agents. Four types of generator properties will be implemented in this simulation model for Prosumer Agents: 

\begin{itemize}
  \item Hydro-electric - a constant source of energy based on a mixture of historical data and projections.
  \item Solar - a source of energy following the typical output profile of a solar panel connected to households.
  \item Wind - a source of energy which will be highly variable in output.
  \item Diesel - a constant source of energy.
\end{itemize}

With the power output of renewable sources of energy such as wind and solar being non-predictable in nature, one of two approach will need to be undertaken to model these sources:
\begin{itemize}
  
  \item A probabilistic generating factor is applied to the generators. This is a constant amount of power each solar panel or wind turbine is assumed to produce during some hours of the day that is to be determined. 
  
  \begin{itemize}
    \item If this method is to be used, the model needs to be implemented in such a way to allow dynamic load-shedding and load-dumping.  
  \end{itemize}
  
  \item A probabilistic generation power output curve based on the least sunny / least windy days
  
  \begin{itemize}
    \item If this method is to be used, sufficient weather and generation data will need to be obtained to implement this method
  \end{itemize}

\end{itemize}

Both methods are in use by distribution networks in the UK for assessing network congestion \cite{IPSA-web-constraint:2015}. \

\paragraph*{Storage} - Storage devices will be batteries of various types that can be connected to the network. For the purpose of this project, it will be assumed that all Prosumer Agents will have one of these to allow allocation of electricity on a hourly basis.

The Storage property should be designed have the capability to prioritise the allocation of its stored energy for certain Agents. For example, the energy output of Storage-only agents could be made to always prioritise the households they are attached to. If the battery box is communal or belongs to a centralised entity such as an e.quinox Energy Kiosk, then no priority will be attached.

\section*{Simulation with Presage 2}
For the initial implementation and testing, the model will only contain two levels of Aggergation (see figure below): \\\\

\begin{center}
Include figure here.
\end{center}

During a round, Agents are expected to publish their Demand/Generation to their Community or Parent Agent. The Parent Agent aggregates this Demand/Generation and publishes to the Supervisor. The Supervisor aggregates the total Demand/Generation and appropriates the electricity fairly.

\section*{Fair Appropriation of the Common Pool Resource}

%% !TEX root = ../main.tex
\chapter{Planning}
\label{Planning}
Following a number of supervisor meetings with Dr Pitt, a number of work items have been identified. A Gantt Chart was created to facilitate the planning and tracking of the project progress. A copy of the current Gantt Chart can be seen in Figure \ref{fig:GanttChart}. The Gantt Chart was a continuous working document which will be update as the project progresses. Tasks were added as new ideas for the project became available. 

\begin{figure}[h!]
\centering
\includegraphics[scale=0.5, angle=90]{Images/GanttChart.png}
\caption{Up to date Gantt chart}
\label{fig:GanttChart}
\end{figure}

Throughout the year, the Gantt Chart was updated to reflect changing priorities, targets and difficulties as the projects progressed. The final project Gantt Chart can be seen below:
\pdfmarkupcomment[markup=Highlight,color=yellow]{ [[[[ "Include Gantt Chart here" ]]]]}{Highlight}
%Include Gantt Chart here
%% !TEX root = ./main.tex
\chapter{Analysis and Design}
\label{analysis}

\subsection{About Presage2}
PRESAGE2 is a simulation platform for multi-nodal or Agent simulation of societies. The platform was built and is currently maintained by PhD students within Imperial. This platform enables the project to investigate the impact of agent design (such as household behaviour), network properties (constraints on access) and the physical environment on individual agent behaviour and long-term global network performance \cite{Presage2-Desc:2015}. In the case of this project, each Node/Agent can represent individuals, households, businesses or generators. 

\subsection{Implementing the Micro Grid in Presage2}
Using Presage2, a network akin to the simplified model in Figure \ref{fig:SimpleModel} will be created and simulated. The Agents are reprsented by the Circles labeled A-H, and the various demand/generation equipment connected on the outside of the network represent the properties the Agents are expected to have.

\begin{figure}[h!]
\centering
\includegraphics[scale=0.8]{Images/Model.png}
\caption{A simplified model diagram}
\label{fig:SimpleModel}
\end{figure}

In the immediate future, it is hoped that this tool could be used to aid the feasibility study of the Micro-Grid to be implemented in Rugaragara Falls. Should there be time, an energy trading platform will be designed and built based on the model created in this project to provide adequate electricity access to all members of the community. However, the energy trading platform would require a different design and implementation scheme which is not included in this report.

The sections below outlines the various properties each Agent must be able to take and how they could be implemented for the simulation scenario of rural communities in Rwanda. All of the properties must be allowed to exist on the same Agent during simulation.

\subsubsection{Agent Properties}

\paragraph{Generator}
Generators will be assumed to be generating power, and not draw any power from the Micro Grid. Four types of generator properties will be implemented in this simulation model: 

\begin{itemize}
  \item Hydro-electric - a constant source of energy based on a mixture of historical data and projections.
  \item Solar - a source of energy following the typical output profile of a solar panel connected to households.
  \item Wind - a source of energy which will be highly variable in output.
  \item Diesel - a constant source of energy.
\end{itemize}

With the power output of renewable sources of energy such as wind and solar being non-predictable in nature, one of two approach will need to be undertaken to model these sources:
\begin{itemize}
  \item A probabilistic generating factor is applied to the generators. This is a constant amount of power each solar panel or wind turbine is assumed to produce during some hours of the day that is to be determined. 
  
  \begin{itemize}
    \item * If this method is to be used, the model needs to be implemented in such a way to allow dynamic load-shedding and load-dumping.  
  \end{itemize}
  
  \item A probabilistic generation power output curve based on the least sunny / least windy days
  \begin{itemize}
    \item * If this method is to be used, sufficient weather and generation data will need to be obtained to implement this method
  \end{itemize}
\end{itemize}

More research will need to be conducted to look at which of the two methods will be more suitable in this application. Both methods are in use by distribution networks in the UK for assessing network congestion \cite{IPSA-web-constraint:2015}.

\paragraph{Demand}
The Demand property will only use electricity in the system. These represent households and businesses in the nearby village. Assumed demand curves will be produced from survey data of potential customers in the area for the initial testing. Should the survey data not be available for the area, a reasonable approximation will be made based a predicted usage habits of the wider local population.

It is anticipated that the final simulation will have a desired demand profile that each Demand-based agent will aim for by trading its allocated energy with other agents. However, the change in demand profile due to trading of energy must be able to satisfy the expected behavioral habits of the local inhabitants. These habits should include restrictions such as no trading during hours of sleep for households. What the habits will be, and how this will be implemented will be determined after some additional research into the area.

\paragraph{Storage}
The Storage property is a dynamic entity which can act as a generator or demand depending on the network utilisation and available power. These storage devices will be batteries of various types that will be connected to the network.

Storage properties will be attached to households which own battery boxes. The Storage property must have the capability to prioritise the allocation of its stored energy for certain Agents. For example, the energy output of Storage-only agents could be made to always prioritise the households they are attached to. If the battery box is communal or belongs to a centralised entity such as an e.quinox Energy Kiosk, then no priority will be attached.

\section{Simulation of the Micro Grid with Presage2}
With the village to be electrified implemented as a model with Presage2, studies will be conducted to determine the best allocation of resources to keep the customers in the village in a reasonably happy state. This would entail that every Agent will be able to consume all of its required energy consumption in the simulation period with a consumption/demand profile which follows all of the properties defined in the Demand property.

Further studies will also be conducted to determine the effects of varying the number of various properties in the network to create for example, a demand-heavy network or a generation-heavy network.  

\subsection{Model Assumptions}
With the Micro Grid model running under Normal Operation, a number of assumptions will be made to simplify the implementation of the model:
\begin{itemize}
\item No losses will be incurred by cables
\item All load on the network will be resistive
\item Only basic appliances such as phones, lights and fridges will be connected to the vast majority of households. A communal fridge-freezer and TV will be available at one of the households
\item Simulations will be run over a variable 1-7 day period
\end{itemize}

\section{Further work}
To facilitate the completion of this project, research will need to be carried out in the following areas:
\begin{itemize}
\item Continual updating of the Gantt Chart to track progress.
\item Research into additional sources of demand and load in rural developing countries and their typical usage to produce suitable realistic generation/demand curves for the simulation.
\item Consumer behaviour of the local population to determine a suitable demand curve, and the properties which are required to be part of the Demand property. e.quinox recently conducted a survey trip in January. The surveys will be used to produce potential demand profiles for the project.
\item Research into electricity generation and distribution infrastructure/equipment available in rural developing countries.
\item Number of households in the village will need to be determined to construct a realistic model.
\item The solar irradiation and photovoltaic cell generation profile for solar panels in developing countries will need to be reliably estimated. Preliminary research suggests an absence of data in this area. Data for Developed countries with similar climates could be used as an estimator.
\item Determine how to best estimate the output of a renewable source of energy dependent on the weather or time of day.
\item Typical storage sizes for battery storage units.
\item Design of Agents in Presage2 for the simulation.
\item Methods for implementing the simulation in Presage2.
\item A variety of rules and other management systems designed to efficiently allocate electricity access.

% !TEX root = ../main.tex
\chapter{Implementation}
\label{Implementation}
In this section, how the design is implemented will be described in detail. 

\section*{Discrete Time Simulation Implementation Overview}
In the simulation, demand and generation requests are made and satisfied on a hourly basis. Due to the nature of Presage 2, actions and requests happen in discrete time steps. In this simulation, each hour has to be split into 5 discrete time-steps due to the parallel and random execution of actions and the discrete time nature of Presage 2. What actions are performed in each of the time steps are outlined below:
\begin{enumerate}
	\item During the first time step, Prosumer Agents submit their generation and demand requests to the Virtual Agent (Community) they are part of.
	\item During the second time step, Virtual Agents (Communities) aggregate the generation and demand requests received and submit their generation and demand requests to the Supervisor Agent.
	\item During the third time step, the Supervisor Agent gathers the total demand and generation requests and subsequently allocates the demand and generation to the Virtual Agents.
	\item During the fourth time step, the Virtual Agents (Communities) receive the allocation given by the supervisor, and allocates that appropriately to the Agents.
	\item During the 5th time step, the Agents appropriates the demand and generation allocation that has been given to them.
\end{enumerate}

\section*{Agent Class Structure and Implementation}
Agents in Presage 2 are created by extending the \textit{AbstractParticipant} class as shown in the \ac{UML} diagram in figure \ref{fig:AgentUML}. 

A Virtual Agent (\textit{ParentAgent} class) represents connected Prosumer Agents. It is required to:
\begin{itemize}
	\item Keep track of the Supervisor Agent, and communicate with it the group demand and generation requests
	\item Keep track of the Prosumer Agents that it is connected to and submit demand and generation requests on their behalf
	\item Submit generation and demand requests to the Supervisor Agent it is connected to
	\item Enforce demand and generation quotas on connected Prosumer Agents
\end{itemize}

To perform the duties outlined above, the following methods are implemented in the \textit{ParentAgent} class:
\begin{itemize}
	\item The \textit{constructor} of this object is responsible for keeping track of which Agent is the Supervisor Agent. The \textit{constructor} is called when this object is created in the initial Simulation set-up as part of the Agent creation and initialisation process.
	\item \textit{addChild()} method is used for keeping track of the Agents. This is also called during the initial Simulation set-up as part of the Agent creation and initialisation process.
	\item \textit{step(int)} method is called by Presage during the simulation to allow the Agent to act on the environment. demand and generation requests are sent to the supervisor, and allocation of demand and generation are also made to the Agents when this method is called automatically by Presage during a simulation.
\end{itemize}

The class \textit{MasterAgent} is how a Supervisor Agent is defined within the Simulator. As a special type of Virtual Agent, it extends from \textit{ParentAgent} class and is required to only do two things:
\begin{itemize}
	\item Keep track of the Virtual Agents in the community it is responsible for
	\item Allocating the correct amount of demand and generation to the Virtual Agents
\end{itemize}

To satisfy those requirements the following methods in the class are implemented:
\begin{itemize}
	\item \textit{addChild()} method is used for keeping track of the Virtual Agents. This is called during the initial Simulation set-up as part of the Agent creation and initialisation process.
	\item \textit{step(int)} method is called by Presage during the simulation to allow the Agent to act on the environment. Demand and generation are allocated to the Virtual Agents when this method is called automatically by Presage
\end{itemize}

Prosumer Agents are Virtual Agents but with no aggregation function, as a Prosumer Agent is the most elementary Agent is the most elementary Agent or sub-community in this holonic system simulation. A Prosumer Agent is required to:
\begin{itemize}
	\item Keep track of which Agent is the Virtual Agent
	\item Submit generation and demand requests to the Virtual Agent it is connected to
	\item Appropriate the allocated amount of generation and demand allocated
\end{itemize}

The Prosumer Agent is implemented with the following methods to allow it to perform the requirements outlined above:
\begin{itemize}
	\item The \textit{constructor} of this object is responsible for keeping track of which Agent is its \textit{ParentAgent}. The \textit{constructor} is called when this object is created in the initial Simulation set-up as part of the Agent creation and initialisation process.
	\item \textit{addChild()} method is used for keeping track of the Agents. This is called during the initial Simulation set-up as part of the Agent creation and initialisation process.
	\item \textit{step(int)} method is called by Presage during the simulation to allow the Agent to act on the environment. demand and generation requests are sent to the supervisor, and allocation of demand and generation are also made to the Agents when this method is called automatically by Presage.
	\item \textit{addProductivity(), addSocialUtility(), addProfileHourly()} are the methods that are called during the initial Simulation set-up as part of the Agent creation and initialisation process to define the properties of this Agent.
\end{itemize}


\begin{figure}[!h]
	\centering
	\includegraphics[scale=0.4]{Images/AgentUML.png}
	\caption{Agent UML Diagram}
	\label{fig:AgentUML}
\end{figure}

\clearpage

\section*{Demand Action}
Agents act on Environments and "Shared States" by performing \textit{Actions}. \textit{Actions} are implemented by extending the Java interface \textit{Action} from Presage 2. In the context of this simulation, an \textit{Action} would be a demand/generation request or a demand/generation allocation. As generation can be modeled as a negative demand, a single \textit{Action} such as the \textit{Demand Action} can be defined to represent both demand requests/quotas and generation requests/dispatch. \textit{childDemand} and \textit{parentDemand} are special instances of \textit{Demand}, and are responsible for representing demand/generation requests from Prosumer Agents and Virtual Agents respectively. The UML diagram of all of the \textit{Actions} in this simulation is defined in the UML diagram in figure \ref{fig:ActionUML}.

\begin{figure}[!h]
	\centering
	\includegraphics[scale=0.5]{Images/ActionUML.png}
	\caption{Actions UML Diagram}
	\label{fig:ActionUML}
\end{figure}

\section*{Action Handlers} % (fold)
\subsection*{Demand Handlers}
To enable the Environment to be able to process the Action requests, \textit{Action Handlers} need to be created to tell the simulation how to deal with Actions from Agents. In Presage 2, \textit{Action Handlers} are created by extending the implementing the Java interface \textit{ActionHandler}. The superclass \textit{Demand} isn't used by any of the Agents, so there are three \textit{Action Handlers}:  \textit{ChildDemandHandler}, \textit{ParentDemandHandler} and \textit{MasterActionHandler}. Depending on which the time step in the simulation hour it is, \textit{ChildDemandHandler} and \textit{ParentDemandHandler} \textit{Action Handlers} store or retrieve requests and allocations accordingly. The \textit{MasterActionHandler} is responsible for aggregating all demand and generation requests from all Virtual Agents and allocating that in time step 3. How these \textit{Action Handlers} relate to each other is described in the UML diagram in figure \ref{fig:ActionHandlerUML}. 

\subsection*{Master Action Handler}
A special case of the \textit{Action Handler} is the \textit{MasterActionHandler}. Unlike the other action handlers which access and data in one localised Environment Services, and methods in another Environment Service to allocate and st, this \textit{Action Handlers} is a \textit{Action-Environment Service} hybrid, which access and stores data in one localised Environment Service

\begin{figure}[!h]
	\centering
	\includegraphics[scale=0.4]{Images/ActionHandlerUML.png}
	\caption{Action Handler UML Diagram}
	\label{fig:ActionHandlerUML}
\end{figure}
% subsubsection subsubsection_name (end)

\clearpage
\section*{Environment Services Implementation}
All Agents perform actions on Environments, which contains one or more "shared states". In Presage 2, all communication that happens between Agents are done via the environment by storing the data in a "shared state". In the context of this simulation, a "shared state" would be information such as demand and generation requests or the amount available in the Common Resource Pool. A visual diagram on how the Environment classes in this simulation are implemented can be found in figures \ref{fig:ServiceUML} and \ref{fig:ServiceUML2}.

\subsection*{Global Environment Service}
All Agents are registered to the Global Environment Service, defined by the \textit{GlobalEnvService} class. Like all Environment service classes in Presage, this was extended from the \textit{EnvironmentService} class built into Presage. This class contains all of the methods that are called when allocations are being made. During time step 3 of a simulated hour, the \textit{allocate()} method is called from the \textit{MasterActionHandler} to perform allocations on behalf of the Supervisor Agent. The demand and generation requests by all of the Virtual Agents are aggregated and stored in this Service. The \textit{allocate()} method by default satisfies all the requirements of connected Virtual Agents if there is enough generation to support it. If there is excess generation, the excess is curtailed proportionally. 
If however, there isn't enough generation, method \textit{allocate\_fairly()} is called, and the Virtual Agents are ranked according to the five applicable \textit{Rescher's Canons of Distributive Justice}. These rankings are computed by calling the methods such as \textit{canon\_of\_equality()} and the allocations are subsequently stored in the \textit{PowerPoolEnvService}.
At the end of the allocation process, the data about the allocation is stored in the environment ready for time step 3 of the next simulated hour via the method \textit{environmentStore}.

\begin{figure}[!h]
	\centering
	\includegraphics[scale=0.45]{Images/EnvironmentUML-1.png}
	\caption{Environment Services UML Diagram}
	\label{fig:ServiceUML}
\end{figure}

\subsection*{Supervisor Environment Service}
The \textit{PowerPoolEnvService} is the Environment Service accessible by both the Supervisor Agent and the Virtual Agents. The \textit{PowerPoolEnvService} extends the \textit{GlobalEnvService} class, and does a lot of the same things on a smaller scale. The \textit{PowerPoolEnvService} is used to store information about the Virtual Agents, such as their aggregated demand and generation requests, and contains the same methods that are called when allocations are being made by the Virtual Agents. During time step 2 of a simulated hour, Virtual Agents sum up their Prosumer Agent demand and generation requests, and store them in \textit{PowerPoolEnvService}. During time step 4 of a simulated hour, the \textit{allocate()} method is called from the \textit{parentDemandHandler}. \textit{allocate()} method by default satisfies all the requirements of Agents if there is enough generation to support it. If there is excess generation, the excess is curtailed proportionally. In a holonic system, Agents are not able to access information concerning other Agents that it is not directly connected to. A separate Environment Service is used to prevent the Supervisor Agent from being able to access and modify "shared states" about Prosumer Agents that are not directly connected to the Supervisor Agent.

Similar to the \textit{GlobalEnvService} class, if there isn't enough generation, method \textit{allocate\_fairly()} is called, and the Agents are ranked according to the five applicable \textit{Rescher's Canons of Distributive Justice}. These rankings are computed by calling the methods such as \textit{canon\_of\_equality()} which are inherited from the \textit{GlobalEnvService}. \textit{PowerPoolEnvService} contains "overriding" methods \textit{allocate()} and \textit{allocate\_fairly()} which do not inherit these from the \textit{GlobalEnvService}, as the allocation data is required to be stored in the \textit{ParentEnvService}. At the end of the allocation process, the data about the allocation is stored in the environment ready for time step 4 of the next simulated hour via the method \textit{environmentStore()}.

\begin{figure}[!h]
	\centering
	\includegraphics[scale=0.5]{Images/EnvironmentUML-2.png}
	\caption{Environment Services UML Diagram}
	\label{fig:ServiceUML2}
\end{figure}

\subsection*{Virtual Agent Environment Service}
The \textit{ParentEnvService} is the Environment Service accessible by both the Virtual Agents and the Prosumer Agents. The \textit{ParentEnvService} inherits from the \textit{PowerPoolEnvService}, and does the same things on a smaller scale. The \textit{ParentEnvService} is used to store information about the Prosumer Agents, such as their individual demand and generation requests, and also contain information about their allocations. During time step 1 of a simulated hour, Agents store their demand and generation requests in this Environment Service as a "shared state". During time step 2 of a simulated hour, Virtual Agents aggregate the stored demand and generation requests and store them in the \textit{PowerPoolEnvService}. During time step 5, Agents retrieve their allocations from the \textit{ParentEnvService}.

\section*{Setting up the simulation}
To set up the simulation with realistic demand and generation profiles, some measured data over a 24 hour period was used to set up the demand and generation profiles of the Agents. To simulate random and independent behaviour between Agents, each of the measured data points was randomly changed by 20\% according to a normal distribution centred around the data points. Figures \ref{fig:WindGenProfile}, \ref{fig:SolarGenProfile} and \ref{fig:DemandProfile} show a plot of the wind generation, solar generation and demand data used to set up the simulation.

\begin{figure} 
	\centering \newlength\figureheight \newlength\figurewidth 
	\setlength\figureheight{6cm} 
	\setlength\figurewidth{13cm} 
	% This file was created by matlab2tikz.
% Minimal pgfplots version: 1.3
%
%The latest updates can be retrieved from
%  http://www.mathworks.com/matlabcentral/fileexchange/22022-matlab2tikz
%where you can also make suggestions and rate matlab2tikz.
%
\definecolor{mycolor1}{rgb}{0.00000,0.44700,0.74100}%
%
\begin{tikzpicture}

\begin{axis}[%
width=0.95092\figurewidth,
height=\figureheight,
at={(0\figurewidth,0\figureheight)},
scale only axis,
xmin=1,
xmax=24,
xlabel={hour},
ymin=0,
ymax=4,
ylabel={Power (kW)},
title style={font=\bfseries},
title={24 hour measured wind generation data}
]
\addplot [color=mycolor1,solid,forget plot]
  table[row sep=crcr]{%
1	1.14\\
2	1.52\\
3	1.28\\
4	1\\
5	1.11\\
6	1.01\\
7	1.64\\
8	2.08\\
9	2.63\\
10	1.97\\
11	2.67\\
12	3\\
13	2.38\\
14	2.61\\
15	1.71\\
16	2.25\\
17	3.12\\
18	3.1\\
19	3.68\\
20	1.94\\
21	3.87\\
22	2.13\\
23	2.37\\
24	2.95\\
};
\end{axis}
\end{tikzpicture}% 
	\caption{Wind Generation Profile} 
	\label{fig:WindGenProfile} 
\end{figure}

\begin{figure} 
	\centering
	\setlength\figureheight{6cm} 
	\setlength\figurewidth{13cm} 
	% This file was created by matlab2tikz.
% Minimal pgfplots version: 1.3
%
%The latest updates can be retrieved from
%  http://www.mathworks.com/matlabcentral/fileexchange/22022-matlab2tikz
%where you can also make suggestions and rate matlab2tikz.
%
\definecolor{mycolor1}{rgb}{0.00000,0.44700,0.74100}%
%
\begin{tikzpicture}

\begin{axis}[%
width=0.95092\figurewidth,
height=\figureheight,
at={(0\figurewidth,0\figureheight)},
scale only axis,
xmin=1,
xmax=24,
xlabel={hour},
ymin=0,
ymax=1.2,
ylabel={Power (kW)},
title style={font=\bfseries},
title={24 hour measured solar generation data}
]
\addplot [color=mycolor1,solid,forget plot]
  table[row sep=crcr]{%
1	0\\
2	0\\
3	0\\
4	0\\
5	0\\
6	0.020833333\\
7	0.069166667\\
8	0.294583333\\
9	0.560833333\\
10	0.7925\\
11	0.962083333\\
12	1.027916667\\
13	1.06125\\
14	1.04625\\
15	1.014166667\\
16	0.818333333\\
17	0.722916667\\
18	0.444583333\\
19	0.205\\
20	0.044583333\\
21	0.004583333\\
22	0\\
23	0\\
24	0\\
};
\end{axis}
\end{tikzpicture}% 
	\caption{Solar Generation Profile} 
	\label{fig:SolarGenProfile} 
\end{figure}

\begin{figure} 
	\centering
	\setlength\figureheight{6cm} 
	\setlength\figurewidth{13cm} 
	% This file was created by matlab2tikz.
% Minimal pgfplots version: 1.3
%
%The latest updates can be retrieved from
%  http://www.mathworks.com/matlabcentral/fileexchange/22022-matlab2tikz
%where you can also make suggestions and rate matlab2tikz.
%
\definecolor{mycolor1}{rgb}{0.00000,0.44700,0.74100}%
%
\begin{tikzpicture}

\begin{axis}[%
width=0.95092\figurewidth,
height=\figureheight,
at={(0\figurewidth,0\figureheight)},
scale only axis,
xmin=1,
xmax=24,
xlabel={hour},
ymin=0,
ymax=3.5,
ylabel={Power (kW)},
title style={font=\bfseries},
title={24 hour measured demand profile data}
]
\addplot [color=mycolor1,solid,forget plot]
  table[row sep=crcr]{%
1	0.35208\\
2	0.96822\\
3	0.83619\\
4	0.48411\\
5	0.35208\\
6	0.30807\\
7	0.92421\\
8	1.40343\\
9	1.00734\\
10	0.74817\\
11	0.79218\\
12	0.8802\\
13	0.96822\\
14	1.31541\\
15	1.22739\\
16	1.431975\\
17	2.773325\\
18	3.033875\\
19	3.033875\\
20	2.512775\\
21	2.58681\\
22	2.10759\\
23	1.01223\\
24	0.4401\\
};
\end{axis}
\end{tikzpicture}% 
	\caption{Demand Profile} 
	\label{fig:DemandProfile} 
\end{figure}

\section*{Issues}
Out of order parallel execution meant that Agents need to submit their individual demands to the SharedState, and have that summed at the end of each time step. It is not possible to sum the demands on the fly.

Being new to both Java and Presage presented problems of its own. It was difficult to understand how simulations could be run and therefore create our own. Some aspects of this project is similar to \textit{LPG'}, but due to API changes the code had to be redesigned and rewritten for this project.

One action per time step meant that it takes 5 time steps to simulate one round of request and appropriation of electricity. It would therefore take 24*4 time steps to simulate a full day of requests and appropriation.

Difficulty with initialising Agents with arrays meant that Agents had to be created with no demand or generation Profiles, and the demand and generation Profiles were added one by one by using the \textit{addProfileHourly()} method.
% !TEX root = ./main.tex
\chapter{Testing}
\label{Testing}
% !TEX root = ./main.tex
\chapter{Results}
\label{Results}
%\include{Chapters/09_evaluation}
% !TEX root = ./main.tex
\chapter{Further Work}
\label{Further Work}

So far, the works carried out will be only be applicable for one community in rural Rwanda. It is hoped that work will be carried out to make this applicable to other communities by including more generation types such as biogas, and more types of appliances owned by each house.

In the immediate future, it is hoped that this tool could be used to aid the feasibility study of the Micro-Grid to be implemented in Rugaragara Falls. Should there be time, an energy trading platform will be designed and built based on the model created in this project to provide adequate electricity access to all members of the community. However, the energy trading platform would require a different design and implementation scheme which is not included in this report.

The sections below outlines the various properties each Agent must be able to take and how they could be implemented for the simulation scenario of rural communities in Rwanda. All of the properties must be allowed to exist on the same Agent during simulation.

In addition, the e.quinox Micro Grid is currently undergoing feasibility studies. If time allows, a system could be designed and built for the Micro Grid to allow the trading of electricity between households to guarantee electricity supply for when it is needed. This system however will have a completely different set of requirements, which will be drawn up after the completion of this project.

Consider Diesel Generators' costs 
Consider satisfaction of individual agents for them to partake in this

In the simulation, carry forward unused generation because we have storage.
% !TEX root = ./main.tex
\chapter{Conclusions}
\label{Conclusions}

Since October, an attempt has been made on understanding the properties of Common Pool Resource, and how electricity can be a Common Pool Resource in the context of a Decentralised Community Energy System. An attempt has been made on building a simulator of a Decentralised Community Energy System in developing countries. 

The simulator demonstrated that the 

The simulation of the Decentralised Community Energy System will be used to explore possible utilisation patterns within rural communities, and also explore controls for regulating the use of electricity. If the study proves to be successful, an attempt at designing and implementing a trading platform for Decentralised Community Energy Systems will be made. 
% !TEX root = ./main.tex
\chapter{User Guide}
\label{Guide}

The github source code repository for this project can be found at: https://github.com/yuchen-w/DCES-Simulator

The code was developed in IntelliJ Idea 14.1.3 IDE, however will work fine on Eclipse. Detailed instructions can be found on the readme of the repository.

% \include{Chapters/chapter1} % Include the first content chapter
 % Include the second content chapter
%\include{Chapters/chapter3} % Include the third content chapter

\backmatter

\chapterstyle{default} % Reset the chapter style back to the default used for non-content chapters

%----------------------------------------------------------------------------------------
%   BIBLIOGRAPHY
%----------------------------------------------------------------------------------------

\bibliographystyle{ieeetr} % Use the plainnat bibliography style

\bibliography{bibliography} % Use the bibliography.bib file as the source of references

%----------------------------------------------------------------------------------------
%   INDEX
%----------------------------------------------------------------------------------------

\printindex % Print the index

%----------------------------------------------------------------------------------------

\end{document}